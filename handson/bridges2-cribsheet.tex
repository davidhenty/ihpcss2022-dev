\documentclass{article}

\usepackage{epcc}

\begin{document}

\title{Crib Sheet: Bridges2 MPI \& OpenMP Exercises}

\author{David Henty}
\date{}
\makeEPCCtitle

%\vspace{-1.0cm}

\section{{\label{sec:access}}Logging on}

Use your username and password to access Bridges2:
\begin{verbatim}
  [user@latop ~]$ ssh -p 2222 xsede-username@bridges2.psc.edu
\end{verbatim}

Although Bridges2 has high performance filesystems which should be
used for large production jobs, here we will just use the home
filesystem for simplicity.

\section{Obtaining source code}

All the course material, including source code, is available on github:

\begin{verbatim}
  [user@bridges2 ~]$ git clone https://github.com/davidhenty/ihpcss2022.git
\end{verbatim}

\section{Compiling code}

For these initial tests, \texttt{cd} into \texttt{hello} and then the
appropriate subdirectory for your chosen language / parallel model.

You have to load a module to access MPI:

\begin{verbatim}
  [user@bridges2 ~]$ module load openmpi
\end{verbatim}

The GNU compilers (\texttt{gcc}, \texttt{g++} and \texttt{gfortran}) are
used for compiling MPI programs via the standard wrappers
\texttt{mpicc}, \texttt{mpicxx} and \texttt{mpif90}.

So, to compile an MPI program in C:

\begin{verbatim}
  [user@bridges2 ~]$ mpicc -o hello hello.c
\end{verbatim}


or in Fortran:

\begin{verbatim}
  [user@bridges2 ~]$ mpif90 -o hello hello.f90
\end{verbatim}

As an interpreted language, there is no explicit compilation stage for
Python programs.

For OpenMP, you pass a special option to the GNU compilers, e.g. for C:

\begin{verbatim}
  [user@bridges2 ~]$ gcc -fopenmp -o hello hello.c
\end{verbatim}

For Fortran:

\begin{verbatim}
  [user@bridges2 ~]$ gfortran -fopenmp -o hello hello.f90
\end{verbatim}

Due to the way that the Python interpreter is implemented, threaded
programming is problematic and there is no Python equivalent of OpenMP.

\section{{\label{sec:running}}Running on Bridges2}

\subsection{MPI}

You cannot run MPI jobs on the login nodes. The simplest approach for
development work on small numbers of processes is to use the
\verb+interact+ command to access the compute nodes.

For example, if you want to run on up to 8 processes:

\begin{verbatim}
  [user@bridges2 ~]$ interact -n 8
\end{verbatim}
    
After a small wait you should be allocated resources and see a
different prompt. You can then run jobs:

\begin{verbatim}
  [user@r001 ~] mpirun -n 8 ./hello
\end{verbatim}

\subsubsection{Python}

Having gained access to the compute nodes using \texttt{interact}, you
have to do some additional setup:

\begin{verbatim}
  [user@r001 ~] module load openmpi  # ??
  [user@r001 ~] module load anaconda3
  [user@r001 ~] pip install mpi4py
\end{verbatim}

Setting up \texttt{mpi4py}, the Python wrappers for the standard MPI
library functions, may take some time when you first install it but
should be much faster afterwards.

You can then run using the standard MPI launcher:

\begin{verbatim}
  [user@r001 ~] mpirun -n 8 python3 ./hello.py
\end{verbatim}

\subsection{OpenMP}

You can run parallel OpenMP jobs on the login nodes, but you should
only do this for small test jobs. For example, to run on 8 threads:

\begin{verbatim}
  [user@bridges2 ~]$ export OMP_NUM_THREADS=8
  [user@bridges2 ~]$ ./hello
\end{verbatim}

For compute intensive jobs, use the \verb+interact+ command to get
access to dedicated resources on the compute nodes as described in the
MPI section above.

\subsubsection{Python}

As mentioned above, there is no equivalent of OpenMP for Python.

\section{{\label{sec:running2}}Running jobs during the Summer School}

For the actual summer school we will have reservations in place to
give you guaranteed access to resources. Instructions on how to access
the reservations will be given at the relevant hands-on sessions.

\end{document}
